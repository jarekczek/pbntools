%{{{ preambule
\documentclass[polish,a4paper,11pt,oneside]{article}
%:encoding=ISO-8859-2:folding=explicit:
\usepackage{html} 
\usepackage{fullpage}
%\usepackage{hyperref}
\usepackage{indentfirst}
\usepackage{parskip}

\sloppy % allow wide spaces between words
% the following adviced by http://sumanta679.wordpress.com/2009/05/20/latex-justify-without-hyphenation/
% to suppress hyphenation while keeping text justified
%\tolerance=1
%\emergencystretch=\maxdimen
%\hyphenpenalty=10000
%\hbadness=10000
%}}}

\include{autoDefs}
\newcommand{\javajdk}{
  Sun Java Development Kit, \htmladdnormallink{JDK 1.6.}
  {http://www.oracle.com/technetwork/java/javase/downloads/index.html}
}
\newcommand{\oldSourceLink}{
  \htmladdnormallink{PbnTools 1.2.5}{http://jarek.katowice.pl/jcwww/pbntools/PbnTools_1_2_5_src.zip}
  , \htmladdnormallink{PbnTools 1.2.4}{http://jarek.katowice.pl/jcwww/pbntools/PbnTools_1_2_4_src.zip}
  , \htmladdnormallink{PbnTools 1.2.3}{http://jarek.katowice.pl/jcwww/pbntools/PbnTools_1_2_3_src.zip}
  , \htmladdnormallink{PbnTools 1.2.2}{http://jarek.katowice.pl/jcwww/pbntools/PbnTools_1_2_2_src.zip}
  , \htmladdnormallink{PbnTools 1.2.1}{http://jarek.katowice.pl/jcwww/pbntools/PbnTools_1_2_1_src.zip}
  , \htmladdnormallink{PbnTools 1.2.0}{http://jarek.katowice.pl/jcwww/pbntools/PbnTools_1_2_0_src.zip}
  , \htmladdnormallink{PbnTools 1.1.0}{http://jarek.katowice.pl/jcwww/pbntools/PbnTools_1_1_0_src.zip}
}
\newcommand{\pbntoolsSourceLinks}{
  \htmladdnormallink{PbnTools \version}{http://jarek.katowice.pl/jcwww/pbntools/PbnTools_\versionUnd\_src.zip}
  , \oldSourceLink
}

\newcommand{\jsoupSourceLinks}{
  % http://jsoup.org/packages/jsoup-1.7.3-sources.jar
  \htmladdnormallink{jsoup-1.7.3-sources.jar}{http://jarek.katowice.pl/jcwww/pbntools/jsoup-1.7.3-sources.jar}
  \htmladdnormallink{jsoup-1.7.1-sources.jar}{http://jarek.katowice.pl/jcwww/pbntools/jsoup-1.7.1-sources.jar}
  \htmladdnormallink{jsoup-1.6.1-sources.jar}{http://jarek.katowice.pl/jcwww/pbntools/jsoup-1.6.1-sources.jar}
}

\newcommand{\bbolink}{
  \htmladdnormallink{BBO}{http://www.bridgebase.com/}
}
\newcommand{\bbolonglink}{
  \htmladdnormallink{Bridge Base Online}{http://www.bridgebase.com/}
}

\newcommand{\sslash}{/}

\newcommand{\zbarOldOrigSourceLinks}{
}
\newcommand{\zbarOldSourceLinks}{
}

\newcommand{\oldBinaries}{
  1.2.5:
  \htmladdnormallink{Windows}{http://jarek.katowice.pl/jcwww/pbntools/PbnTools\_1\_2\_5\_win.zip}
  \htmladdnormallink{Linux}{http://jarek.katowice.pl/jcwww/pbntools/PbnTools\_1\_2\_5\_linux.zip}
  , 1.2.4:
  \htmladdnormallink{Windows}{http://jarek.katowice.pl/jcwww/pbntools/PbnTools\_1\_2\_4\_win.zip}
  \htmladdnormallink{Linux}{http://jarek.katowice.pl/jcwww/pbntools/PbnTools\_1\_2\_4\_linux.zip}
  , 1.2.3:
  \htmladdnormallink{Windows}{http://jarek.katowice.pl/jcwww/pbntools/PbnTools\_1\_2\_3\_win.zip}
  \htmladdnormallink{Linux}{http://jarek.katowice.pl/jcwww/pbntools/PbnTools\_1\_2\_3\_linux.zip}
  , 1.2.2:
  \htmladdnormallink{Windows}{http://jarek.katowice.pl/jcwww/pbntools/PbnTools\_1\_2\_2\_win.zip}
  \htmladdnormallink{Linux}{http://jarek.katowice.pl/jcwww/pbntools/PbnTools\_1\_2\_2\_linux.zip}
  , 1.2.1:
  \htmladdnormallink{Windows}{http://jarek.katowice.pl/jcwww/pbntools/PbnTools\_1\_2\_1\_win.zip}
  \htmladdnormallink{Linux}{http://jarek.katowice.pl/jcwww/pbntools/PbnTools\_1\_2\_1\_linux.zip}
  , 1.2.0:
  \htmladdnormallink{Windows}{http://jarek.katowice.pl/jcwww/pbntools/PbnTools\_1\_2\_0\_win.zip}
  \htmladdnormallink{Linux}{http://jarek.katowice.pl/jcwww/pbntools/PbnTools\_1\_2\_0\_linux.zip}
  , 1.1.0:
  \htmladdnormallink{Windows}{http://jarek.katowice.pl/jcwww/pbntools/PbnTools\_1\_1\_0\_win.zip}
  \htmladdnormallink{Linux}{http://jarek.katowice.pl/jcwww/pbntools/PbnTools\_1\_1\_0\_linux.zip}
}

\newcommand{\bugsonetwozero}{
  \bug{52}, \bug{53}, \bug{54}, \bug{55}, \bug{56},
  \bug{57}, \bug{58}, \bug{59}, \bug{60}
}

\newcommand{\bug}[1]{
  \htmladdnormallink{#1}{http://jarek.katowice.pl/other/bugzilla/show_bug.cgi?id=#1}
}


\begin{document}
\title{PbnTools - help}
\author{\null}
\date{\null}


\maketitle

This is a shortened help. Full version available in \htmladdnormallink{Polish}{help_pl.html}.

\tableofcontents

%{{{ intro
\section{Introduction}
\htmladdnormallink{PbnTools}{http://jarek.katowice.pl/pbntools-en} is intended as a toolkit for files containing bridge deals.
Most popular format for these is \htmladdnormallink{PBN}{http://www.tistis.nl/pbn/pbn_v21.txt}.

Currently PbnTools has following funcionalities:

\begin{description}
\item[Download Kops]
  Downloads tournaments in kops format and saves it as PBN files.
\item[Deal cards]
  Help in manual duplication of bridge boards from pbn files.
  Requires an internet camera and barcoded playing cards.
\end{description}

Program is started with the file PbnTools.exe (Windows) or pbntools (linux).
%}}}

%{{{ license
\section{License}
The program is distributed together with source code
under \htmladdnormallink{GNU~GPL}{gpl30.html} license,
version 3 or later.

\subsection{Attached software}

The program uses following software:

\begin{description}
\item[JLayer]
  \htmladdnormallink{JLayer}{http://www.javazoom.net/javalayer/javalayer.html}
  is a JAVA\textsuperscript{TM} library that decodes, converts and plays MP3 files in real-time.
  Distributed under LGPL license.
  See files {\em jlayer\_LICENSE.txt} and {\em jlayer\_README.txt} in program directory.

\item[ZBar]
  \htmladdnormallink{ZBar}{http://zbar.sourceforge.net/}
  is an open source software suite for reading bar codes from various sources,
  such as video streams, image files and raw intensity sensors.
  Distributed under LGPL 2.1 (or later) license.
  Here is applied a limited version reading pictures from a video camera.
  A license file: zbar\_LICENSE.txt

\end{description}

Windows version uses also:
  
\begin{description}

\item[msys]
  Linux scripting environment working under Windows, thus mainly {\em bash}.
  Part of files were used, from version
  \htmladdnormallink{1.0.11}{https://sourceforge.net/projects/mingw/files/MSYS/BaseSystem/msys-core/msys-1.0.11/}.
  Complete binary suite:  
  \htmladdnormallink{msysCORE-1.0.11-bin}{http://jarek.katowice.pl/jcwww/pbntools/msysCORE-1.0.11-bin.tar.gz},
  \htmladdnormallink{wget-1.12-1-msys-1.0.13-bin.tar.lzma}{http://jarek.katowice.pl/jcwww/pbntools/wget-1.12-1-msys-1.0.13-bin.tar.lzma}.
  
  Msys being a suite of many programs has not a single license.
  This is described on
  \htmladdnormallink{msys}{http://www.mingw.org/wiki/MSYS} page,
  in section \htmladdnormallink{Licensing Terms}{http://www.mingw.org/%
  license}.
  After downloading source, one can find the licenses for all the components.
\item[wget]
  \htmladdnormallink{wget for windows}{http://gnuwin32.sourceforge.net/packages/wget.htm},
  belonging to \htmladdnormallink{GnuWin32}{http://gnuwin32.sourceforge.net/},
  is a tool for recursive download of web pages, together with all needed files.
  License file: wget\_COPYING.txt.
  \htmladdnormallink{binaries}{http://downloads.sourceforge.net/gnuwin32/wget-1.11.4-1-bin.zip},
  \htmladdnormallink{dependencies}{http://downloads.sourceforge.net/gnuwin32/wget-1.11.4-1-dep.zip}
  
\end{description}

Source codes are available in section Download (\ref{downloadSrc}).
%}}}

%{{{ functionality
\section{Functionality}
%{{{ download kops
\subsection{Download Kops} \label{pobKops}

Polish tournaments published in \htmladdnormallink{Kops}{http://jfr.pzbs.pl/}
format can be found on pages of Polish Contract Bridge Association
(\htmladdnormallink{PZBS}{http://www.pzbs.pl}).
Here are \htmladdnormallink{sample results}{http://www.slzbs.pl/index.php?main=6}.
The best tournaments are labeled {\em Chorz�w}.

Program requires a link to a tournament.
It should be a link labeled {\em Wyniki},
leading to a screen consisting of 2 frames: results and boards.
Generated files are inserted into working directory ({\em Configuration}),
in a subdirectory {\em kops}. Subdirectory is generated automatically.
%}}}

%{{{ deal cards
\subsection{Deal cards} \label{rozdaj}

Dealing cards is done by showing cards to an internet camera,
which "finds" the card in a given board and points to whom it should be dealt.
Pointing is done by displaying player's sign and speaking it through the speaker.
Currently Polish words are used: {\em jeden, dwa, trzy, cztery},
respectively for {\em N, E, S, W}.

\subsubsection{Requirements}
\begin{description}
\item[Internet video camera]
  The best in this case are cameras with manual focus adjustment (with a knob).
  
\item[Karty z kodami paskowymi]
  Cards licensed by \htmladdnormallink{Jannersten}{http://www.jannersten.se/}
  should be used, format \htmladdnormallink{WIN}{http://www.jannersten.se/html/win.html}.
\end{description}

\subsection{Configuration} \label{konfig}

\begin{description}
\item[Working directory]
\item[Zbarcam options]
  Additional options passed to zbarcam.
  A video device may be specified this way as e.g. \verb!/dev/video1!.
  Default (first) device has number 0.
  Full options list is available in file {\em zbarcam.html}.

\end{description}

%}}}
%}}}


%{{{ requirements
\section{Requirements}

Java, version 6, is required. Platforms supported: Windows and Linux.
%}}}

%{{{ download
\section{Download} \label{download}

\subsection{Installable versions} \label{downloadBin}

Current version, \version{}:

\begin{itemize}
\item
  \htmladdnormallink{Windows}{http://jarek.katowice.pl/jcwww/pbntools/PbnTools_\versionUnd\_win.zip}
\item
  \htmladdnormallink{Linux}{http://jarek.katowice.pl/jcwww/pbntools/PbnTools_\versionUnd\_linux.zip}

\end{itemize}

\subsection{Source code} \label{downloadSrc}

\begin{itemize}
\item
  \htmladdnormallink{PbnTools \version}{http://jarek.katowice.pl/jcwww/pbntools/PbnTools_\versionUnd\_src.zip}
  or see {\em Development}, section \ref{dev}
\item
  \htmladdnormallink{jlayer1.0.1.tar.gz}{http://jarek.katowice.pl/jcwww/pbntools/jlayer1.0.1.tar.gz}
\item
  ZBar
  \begin{itemize}
  \item
    \htmladdnormallink{zbar original}{http://jarek.katowice.pl/jcwww/pbntools/%
    zbar\_src\_orig.tgz}
  \item
    \htmladdnormallink{zbar modified}{http://jarek.katowice.pl/jcwww/pbntools/%
    zbar\_src\_pbntools.tgz}
  \item
    A patch set available with PbnTools source code.
  \end{itemize}

\item
  \htmladdnormallink{msys}{http://jarek.katowice.pl/jcwww/pbntools/msysCORE-1.0.11-src.tar.gz}
  
\item
  \htmladdnormallink{wget for windows}{http://jarek.katowice.pl/jcwww/pbntools/wget-1.11.4-1-src.zip}
  \htmladdnormallink{wget for msys}{http://jarek.katowice.pl/jcwww/pbntools/wget-1.12-1-msys-1.0.13-src.tar.lzma}
  

\end{itemize}

\subsection{Installation}
Unpack zip file into an arbitrary target folder.
This folder may not contain spaces.
%}}}

%{{{ rozw�j
\section{Development} \label{dev}

Program is written mainly in Java.
I encourage to contribute either by programming or by submitting bug reports and suggestions.

Following development means are available:

\begin{itemize}
\item
  \htmladdnormallink{Bugzilla}{http://jarek.katowice.pl/bugzilla} -
  submit bugs and suggestions.
  
\item
  \htmladdnormallink{Dedicated forum}{http://jarek.katowice.pl/%
  other/forum/viewforum.php?f=10} -
  discussion.
  
\item
  SVN - source code, thanks to 
  \htmladdnormallink{sourceforge.net}{https://sourceforge.net}:
  \begin{verbatim}
    svn checkout svn://svn.code.sf.net/p/pbntools/code/trunk pbntools-code-directory
  \end{verbatim}

  \htmladdnormallink{Browse code on-line}{https://sourceforge.net/p/pbntools/%
  code/}, 
  \htmladdnormallink{PbnTools at sourceforge}{https://sourceforge.net/projects/pbntools}, 

\end{itemize}

Guidelines for source code are in file README.SRC.txt,
in source code archive, see: Download, \ref{download}.

\subsection{Build requirements}

\begin{enumerate}
\item
  \javajdk
\item
  \htmladdnormallink{ant}{http://ant.apache.org/}, Debian package name: ant.
  Main build tool.
\item
  \htmladdnormallink{BeanShell}{http://www.beanshell.org/},
  {\em bsh-2.0b4.jar} (or newer) pointed in {\em build.xml}
%\item
  %\htmladdnormallink{makeself}{http://megastep.org/makeself/}, nazwa pakietu w Debianie: makeself.
  %Do zbudowania wersji dystrybucyjnej pod linuxa.
\item
  For building zbar additional tools are needed.
  Building zbar is not necessary.
\item
  Latex tools for documentation. Debian package name: latex2html.
\end{enumerate}
%}}}

%{{{ history
\section{Version history}
\begin{description}
\item[1.0.0, 01.11.2011]
  First official release, under GNU GPL 3.0.
  \begin{itemize}
  \item Option {\em Deal cards}
  \end{itemize}
\item[0.5, 07.03.2010]
  Private version.
  \begin{itemize}
  \item Option {\em Download Kops}.
  \end{itemize}
\end{description}
%}}}

%{{{ zakonczenie
\section{Ending}

I am open for suggestions. I will be grateful for informations about incorrect operation of the program or a web-page.

{\setlength{\parindent}{0pt}\small\htmladdnormallink{Jarek Czekalski}{mailto:jarekczek@poczta.onet.pl}, last modification Oct 30 2011}
%}}}




\end{document}
